\documentclass[twoside, 12pt]{article}
\usepackage{xdipp}
\begin{document}
\titul{Návrh nasazení on-premises privátního cloudu Mendelovy univerzity v Brně}{Adam Šimek}{Ing. Jiří Passinger}{Brno~2024}
\podekovani{Bude doplěno po dokončení práce.}
\prohlasenimuz{V Brně dne 20. února 2024}
\abstract{Šimek, A. Design of on-premises private cloud deployment on Mendel University in Brno. Brno: Mendel University in Brno, 2024.}{Will be added after the completion of thesis.}
\abstrakt{Šimek, A. Návrh nasazení on-premises privátního cloudu Mendelovy univerzity v Brně. Brno: Mendelova Univerzita v Brně, 2024.}{Bude doplněn po dokončení práce.}
\klslova{Virtualizace, CloudStack, OpenStack, OpenNebula, privátní cloud, IaaS}
\keywords{Virtualization, CloudStack, OpenStack, OpenNebula, private cloud, IaaS}
\obsah
\listoffigures
\listoftables
\kapitola{Úvod a cíl práce}
\sekce{Úvod}
Mendelovu univerzitu v Brně lze považovat za velkou organizaci, které je ve velké míře závislá na informačních technologií a bylo by vhodné, aby po vzoru podobně rozsáhlých organizacích zaměřených na IT nasadila svůj privátní cloud, který by mohl zkvalitní a zjednodušit výuku. Hlavní problémy, které by privátní cloud vyřešil jsou softwarové nedostatky při výuce zejména předmětu zaměřených na IT. Vyřešena by tak byla problematika spojení virtualizace a kontejnerizace na jenom fyzickém stroji s omezenými oprávněními tak že celé laboratorního prostředí bude přemístěno z počítačů v učebně na více flexibilní cloudovou platformu ke které budou mít studenti ať už kombinované nebo prezenční formy studia odkudkoliv neustálý přístup a budou ho tak navíc schopni využívat i pro samostudium. Vyučující si budou schopni přizpůsobit laboratorní prostředí pomocí intuitivního rozhraní nebo procesu IaC. Nyní také chybí studentům produkční prostředí, kde by své aplikace mohli nasadit a ladit, v univerzitní cloudu by si tak pro tyto účely mohli vytvořit vlastí virtuální server. Cloud by také umožňoval provozování různých ať už existujících nebo budoucích IT služeb univerzity, takže jak výukovou, tak i produkční virtualizaci by bylo možné spravovat jednotně ve stejném cloudovém prostředí.
\sekce{Cíl práce}
Výběr a testovací nasazení vhodného on-premise privátního cloudu  který bude splňovat stanovené požadavky. Výsledkem výběru bude dostatečně robustní škálovatelný privátní cloud, který by mohli využívat vyučující pro výuku předmětů ve kterých je potřeba virtuální laboratorní prostředí, studenti jej navíc budou moc využít pro samostudium a v neposlední řadě také samotná univerzita pro provoz svých služeb.
\kapitola{Metodika}
V teoretické etapě práce zpracuji z veřejně dostupných zdrojů rešerši, ve které se budu zaobírat problematikou virtualizace, kontejnerizace a tvorbou privátního cloudu, následně od pověřené osoby z OIT zjistím, jaké požadavky na cloud mají a co od něj očekávají. Na základě zjištěných informací vyberu několik cloudových implementací, které mezi sebou teoreticky porovnám. Následovat bude etapa praktická ve které nasadím řešení, která byla vybrána v předchozí etapě a posoudím jejich praktickou využitelnost v reálném prostředí. Výsledkem těchto dvou etap bude vybráno jedno řešení, které vyhovuje stanoveným kritérium a je vhodné a zároveň přínosné pro nasazení do univerzitního prostředí.
\kapitola{Rešerše}
Pro zpracování rešerše byly vybírány jak zdroje  zabývají problematikou privátních cloudů jak v praktické tak teoretické rovině. 

Pro teorii jsou vybrány knihy ve kterých je do hloubky popsáno co privátní cloud znamená, jak ho navrhnout a udržovat. Konkrétní publikace byly vybrány na základě doporučení uvedených na webových stránkách daných platforem. 

Za účelem získání informací o reálném využití této technologie v oblasti vzdělávacích institucí byly zvoleny závěrečné práce studentů, kteří je pro své univerzity ve svých pracích zkoumali a vyzkoušeli. Tyto práce byly dohledány pomocí vyhledávače Google Scholar. Pro získání informací z produkčního prostředí byly z kanálů konferencích na online video platformě YouTube vybrány nahrávky, kde studenti nebo vyučující prezentují své nasazené univerzitní cloudy. Za účelem posouzeni výkonu jednotlivých řešení a hlubšímu technickému porozumění byly vybrány odborné články z konferencí které jsou dostupné v digitální knihovně IEEE Xplore.
\sekce{Knihy}
\textbf{OpenStack for Architects - Second Edition} \\
Kniha která se do detailně zaměřuje na jednu z platforem pro správu cloudu, která je v tomto případě OpenStack. Na začátku autor popisuje zejména jednotlivé komponenty ze kterých se tento software skládá. V několika dalších kapitolách je vysvětleno na co by se měl architekt při návrhu privátního cloudu zaměřit. Ke každé z kapitol je také uveden praktický příklad
V této knize autor nejprve shrne současný stav OpenStacku a jeho jednotlivých komponent. Následně na praktických příkladech popisuje životní cyklus privátního cloudu od návrhu až po day 2 správu. Popsáno je také jak nově vytvořený privátní cloud začlenit do stávají podnikové infrastruktury a také jak na něj aplikovat metodiku DevOps. Vynechány nejsou ani postupy jak privátní cloud správně zabezpečit. \cite{architects} Kniha mi detailně popsala z jakých komponent se platforma OpenStack skládá a jak jsou spolu provázané. Za její největší přínos ale považuji že jsem se dozvěděl jak by měla vypadat architektura privátní cloudu z pohledu výpočetní síly, úložiště a síťování.\\
\textbf{Apache CloudStack Cloud Computing} \\
Jedná se o knihu ve které se autoři zaměřili na jednu z platforem pro správu cloud známou jako CloudStack. Nejprve je čtenář krátce seznámen světem cloudů a se zdroji které může poskytovat. Následně jsou představeny a vysvětleny koncepty které jsou specifické pouze pro platformu CloudStack. V dalších kapitolách je popsán postup instalace a tvorba clusteru spolu s minimální nutnou konfigurací. Detailně jsou také popsány možnosti síťování na této konkrétní platformě. Dál jsou popsány administrativní činnosti týkající jak virtuálních strojů tak samotných uživatelských účtů. Na závěr je také vysvětlena implementace vysoké dostupnosti spolu s možnostmi škálování. \cite{apachebook} Kniha mi pomohla pochopit důležité základní a pokročilé koncepty této platformy, které jsou neměnné napříč verzemi. Jednalo se tak především o koncepty zón, úložiště a síti. Ovšem metody instalace a konfigurace jsou v dnešní době už jiné proto mi i přes to že byly dobře popsány nebyly přínosné.  
\sekce{Závěrečné práce}
\textbf{Designing and implementing a private cloud for student and faculty software projects}\\
V této závěrečné práci se autoři věnují výběru vhodného softwaru privátního cloudu pro potřeby jejich univerzitu. Nejprve jsou popsány rozdíly mezi veřejnými, privátními, hybridními a komunitními cloudy a kategoriemi služeb IaaS, PaaS, SaaS a CaaS. Na základě rozhovorů se studenty a akademickými pracovníky vyberou autoři tří implementace privátního cloud které povrchově popíší  a nasadí na testovací infrastruktuře. Následně u každé z implementací rozeberou s jakými problémy se potýkali a jaký z ní mají pocit. Na závěr implementaci která splnila nejvíc stanovených požadavků univerzity zevrubně popíši, automatizují tvorbu instancí, které doplní o GPU akceleraci. \cite{kth} Ze závěrečné práce jsem zjistil jaké open-source platformy pro správu cloudu existují a jaké jsou mezi nimi základní rozdíly. Dál jsem z ní také pochopil základní rozdíly mezi as-a-service službami. Ovšem největší přínos závěrečné práce bylo pro mě to, že jsem ze dozvěděl na které otázky bych se měl při konzultacích s OIT zaměřit.\\
\textbf{Evaluation of a Private Cloud for Higher Education} \\
Autor se v této závěrečné práci zabývá nasazením privátního cloudu který by sloužil pro výuku předmětů zaměřených na informační technologie. Na začátku práce je čtenáři vysvětlena historie virtualizace a základní pojmy z oblasti cloud computingu. Za účelem stanoveni požadavků které by měl cloud splňovat autor provedl několik rozhovorů na základě kterých stanovil několik primárních a sekundárních funkcí které jsou od cloudu očekávány. Následuje teoretická kapitola ve které jsou porovnány různé platformy pro správu cloudu a vybrán je platforma OpenStack. V praktické části je popsáno testovací prostředí z pohledu hardwaru, síťování, hypervizora a úložiště. Nastíněna je také konfigurace. Zmíněny jsou také rutinní činnosti zálohování a systémové údržba. Na konci této části je prozkoumáno zda byly splněny požadavky které byly pro privátní cloud stanoveny. Na závěr jsou také prezentována a vyhodnocena data ze čtyř měsíčního testovacího nasazení. \cite{trygve} Z této práce jsem se především dozvěděl jak si efektivně stanovit a zpracovat požadavky, které budou uplatněny při výběru platformy a jak pro tento proces vizualizovat pomocí jazyka UML. V neposlední řadě jsem si také osvěžil historii a základní principy virtualizace. 
\sekce{Odborné články}
\textbf{Private IaaS Clouds: A Comparative Analysis of OpenNebula, CloudStack and OpenStack}\\
Odborný článek který se zabývá hodnocením pružnosti a flexibility platforem pro správu cloudu. Autoři vytvořili a popsali vlastní metodiku hodnocení dvou zmíněných kategorií, kterou aplikovali na konkrétní verze OpenStack, CloudStack a OpenNebula. Dál je také na základě standardizovaných testů změřen a porovnán výkon jednotlivých řešení. Výstupem je tak vítěz pro každou z těchto oblastí. Na závěr jsou také zmíněny společné oblasti pro zlepšení. \cite{analysis} Článek mi pomohl zjistit jaké jsou silné a slabé stránky jednotlivých platforem ve společně posuzovaných oblastech. Také jsem se dozvěděl z jakých komponent se jednotlivé funkcionality platforem skládají.\\
\textbf{Deploying an OpenStack cloud computing framework for university campus}\\
V tomto odborném článku se čtenář dozví o nasazení malého privátního cloudu na jedné z indických univerzit. V úvodu článku je popsáno proč nabývá cloud computing na popularitě a jaké jsou jeho výhody pro použití na univerzitách. V části literárního přehledu jsou vyjmenovány vzdělávací instituce, které řešení cloud computing úspěšně implementovali a používají ji pro účely výuky. V další části je popsáno jak univerzita plánuje svůj on-premises privátní cloud nasadit a k jakým účelům je plánováno jej využívat. Samotné zprovoznění privátního cloudu v praxi je popsáno v implementační části článku, kde je stanoveno jaký bude použit hardware a jaké typy uzlů z platformy OpenStack budou nasazeny. Pro otestování funkčnosti je také spuštěna testovací instance odlehčeného operačního systému. V závěru je zmíněno o jaké další funkce chce univerzita svůj nový cloud rozšířit. \cite{india} Z článku jsem se dozvěděl kromě toho jaké jsou výhody cloud computing také, které univerzity tuto možnost pro svoji výuku již významně používají a jaké u nich našla využití. Dozvěděl jsem se i o dalších možnostech jak privátní cloud využívat. Také jsem zjistil jaká pravidla jsou z hlediska cloudu důležitá pro distanční výuku a o konceptu LaaS.\\
\textbf{}
\textbf{Infrastructure as a service (IaaS): A comparative performance analysis of open-source cloud platforms}
*\cite{comparison}*
\sekce{Přednášky}
\textbf{Building a Software Makerspace with CloudStack to Drive Innovation}\\
Na této přednášce prezentují studenti produkční univerzitní cloud založený na platformě CloudStack. Nejprve se od posluchači dozvědí že k vytvoření cloud vedla studenty vize že i oni by měli mít vlastní dílny na tvoření, jako u jiných odvětví. Dál je zmíněno že nyní tento cloud nepoužívají pouze studenti ale i vyučují pro výuku a výzkumní pracovníci, možné je že v budoucnu ho budou používat i jiné instituce. V další části přednášky je prezentován také zjednodušený přehled z jakých dílčích částí se cloud skládá. Na závěr se dozvídáme s jakými problémy se studenti setkali a jaké mají s cloudem další plány, zmíněno je také jaké vylepšení by si od této platformy přáli. \cite{cloudstackcon} Z přednášky jsem se dozvěděl proč je pro studenty důležité aby měli vlastní prostředí pro testování a že v některých případech může být pro uživatele vhodné vytvořit vytvořit vlastní front-end. Také jsem se zjistil jaké jsou obvyklé netechnické problémy při nasazování a údržbě privátního cloudu ve veřejném sektoru.\\
\textbf{The U.S. Army Cyber School OpenStack Use Case}\\
Tato přednáška pojedná o využití privátního cloudu v oblasti výuky kyberbezpečnosti. Nejdříve jsou vymezeny standardy, kterými se výuka musí řídit a tím pádem jsou zmíněny i metody agilní výuky. Na tuto myšlenku navazuje popsaní přechodu ze zastaralého vývoje kurzů na nový, který je založený na metodách DevOps. Tím pádem došlo k vytvoření systému založeném na těchto konceptech fungující na platformě OpenStack a dostupný i mimo místo výuky. Popsány jsou také začátky na pár serverech, které se díky velké úspěšnosti projektu během roku rozrostli na plnohodnotné datové centrum. V návaznosti na toto téma jsou zmíněny i důležité administrativní problémy se kterými se tým administrátorů potýkal a jaké si odnesli ponaučení. Ukázány jsou také automatizační skripty, které jsou open-source a jsou schopny od nuly vytvořit produkční infrastrukturu. \cite{usarmy} Díky přednášce jsem zjistil jaký je přínos uplatnění DevOps při tvorbě a výuce kurzů. Dozvěděl jsem se také jak moc může být pro malý tým Git a proces IaC užitečný. Uvědomil jsem si také jak moc je důležitá spolupráce všech zúčastněných stran při implementaci nového systému. 
\kapitola{Analýza problému}
\sekce{Počáteční stav}
\textbf{Výuka}\\
V době psaní této bakalářské práce se předměty zaměřené na seznámení studentů s operačními systémy vyučují následujícími způsoby. Na každém počítači v učebně je nainstalován operačním systém Windows 10 ve kterém se aktivována funkcionalita Hyper-V, pomocí které je pro každý předmět vytvořen požadovaný počet virtuálních strojů, které jsou spolu v závislosti na předmětu propojeny za účelem simulace rozsáhlejší sítě. Každý virtuální stroj má zpravidla jeden nebo více snapshotů různých výchozích stavů v závislosti na sylabu předmětu. Pro předmět ve kterém se probírají databázové systémy se používá Docker Desktop. (poznámka pod čarou) V tomto případě jsou jednotlivé stažené image a z nich vytvořené kontejnery odděleny podle uživatelských účtů hostitele.
\sekce{Požadovaný stav}


\kapitola{Použité technologie}
\begin{literatura}
\citace{kth}{Le Fevre, 2022}
{\autor{LE FEVRE, Pierre a Emil KARLSSON}. \nazev{Designing and implementing a private cloud for student and faculty software projects} [online]. 2022 [cit. 2024-02-23]. Dostupné z: https://kth.diva-portal.org/smash/get/diva2:1666025/FULLTEXT01.pdf}
\citace{usarmy}{Rodriguez, 9. 5. 2017}
{\autor{RODRIGUEZ, Julianna a Christopher APSEY}. The U.S. Army Cyber School OpenStack Use Case [přednáška]. In: \nazev{OpenInfra Summit Boston 2017}. Boston, OpenInfra Foundation, 8. 5. 2017 [cit. 2024-02-26]. Dostupné z: https://youtu.be/fRRiQVxbQ1g}
\citace{apachebook}{Sabharwal, 2013}
{\autor{SABHARWAL, Navin a Ravi SHANKAR}. \nazev{Apache CloudStack Cloud Computing}. Birmingham: Packt, 2013. ISBN 978-1-78216-010-6.}
\citace{comparison}{Shahzadi, 2017}
{\autor{SHAHZADI, Sonia} et al. \nazev{Infrastructure as a service (IaaS): A comparative performance analysis of open-source cloud platforms} [online]. 2017 [cit. 2024-03-01]. Dostupné z: https://ieeexplore.ieee.org/document/8031522}
\citace{india}{Sheela, 2017}
{\autor{SHEELA, P. a Monika CHOUDHARY}. \nazev{Deploying an OpenStack cloud computing framework for university campus} [online]. 2017 [cit. 2024-03-01]. Dostupné z: https://ieeexplore.ieee.org/document/8229908}
\citace{architects}{Silverman, 2018}
{\autor{SILVERMAN, Ben a Michael SOLBERG}. \nazev{OpenStack for Architects}. Birmingham: Packt, 2018. 2nd ed.. ISBN 978-1788624510.}
\citace{trygve}{Tønnesland, 2013}
{\autor{TØNNESLAND, Trygve}. \nazev{Evaluation of a Private Cloud for Higher Education} [online]. 2013 [cit. 2024-02-25]. Dostupné z: https://ntnuopen.ntnu.no/ntnu-xmlui/bitstream/handle/11250/253239/648705\_FULLTEXT01.pdf?sequence=1\&isAllowed=y}
\citace{analysis}{Vogel, 2016}
{\autor{VOGEL, Adriano}. \nazev{Private IaaS Clouds: A Comparative Analysis of OpenNebula, CloudStack and OpenStack} [online]. 2016 
[cit. 2024-02-22]. Dostupné z: https://ieeexplore.ieee.org/document/7445407}
\citace{cloudstackcon}{Willen, 23. 11. 2023}
{\autor{WILLEN, Jonas, Pierre LE FEVRE a Emil KARLSSON}. Building a Software Makerspace with CloudStack to Drive Innovation [přednáška]. In: \nazev{CloudStack Collaboration Conference 2023}. Paříž, Apache CloudStack Community, 23. 11. 2023 [cit. 2024-02-26]. Dostupné z: https://youtu.be/QYT1HA6krU4}
\end{literatura}
\end{document}